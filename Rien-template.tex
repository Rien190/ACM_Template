\documentclass[a4paper,11pt]{article}
\usepackage{zh_CN-Adobefonts_external} % Simplified Chinese Support using external fonts (./fonts/zh_CN-Adobe/)
\usepackage{fancyhdr}  % 页眉页脚
\usepackage{minted}    % 代码高亮
\usepackage[colorlinks]{hyperref}  % 目录可跳转
\usepackage{graphicx}
\setlength{\headheight}{15pt}

\hypersetup {
    colorlinks=true,
    linkcolor=black
}
\renewcommand\theFancyVerbLine{\large\arabic{FancyVerbLine}}

% 定义页眉页脚
\pagestyle{fancy}
\fancyhf{}
\fancyhead[C]{ACM Template by Rien}
\lfoot{}
\cfoot{\thepage}
\rfoot{}

\author{Rien}
\title{ACM Template}

\begin{document}
\maketitle % 封面

\begin{figure}[H]
    \centering
    \includegraphics[width=0.45\textwidth,]{picture/logo.jpg}
    \vspace{0.5cm}
\end{figure}
\centerline{rien\_zhu@163.com}

\newpage % 换页
\tableofcontents % 目录
%\twocolumn  % 分页显示
\newpage
\section{字符串}
\subsection{KMP}
\inputminted[breaklines,linenos,frame=leftline]{c++}{string/kmp.cc}

\subsection{Suffix Automaton}
\inputminted[breaklines]{c++}{string/suffix-automaton.cc}

\newpage
\section{图论} % 一级标题
\subsection{Minimum Spanning Tree} % 二级标题
\subsubsection{Kruskal} % 三级标题
\inputminted[breaklines]{c++}{graph/kruskal.cc} % 插入代码文件
% 中文测试
\subsection{单源最短路}
\subsubsection{SPFA}
\inputminted[breaklines]{c++}{graph/spfa.cc}
\subsection{我也不知道}

\section{其他}

\end{document}
